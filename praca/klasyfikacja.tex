\chapter{Klasyfikacja danych}

 Człowiek od zawsze próbuje skategoryzować i uporządkować posiadaną wiedzę w zbiory pozwalające na łatwy dostęp do przechowywanych informacji na przykład w sposób tabelaryczny.\ Klasyfikacja to proces rozpoznawania obiektów na bazie analizy ich cech.\ Jest to jedna z pierwszych rzeczy, której uczą się niemowlęta: rozróżniania kształtów, kolorów czy własnych rodziców. \ W póżniejszych latach te umiejętności są bardzo ważne, ponieważ w otaczającym świecie wiele rzeczy jest sklasyfikowanych.\ Np książki w katalogach bibliotecznych czy produkty spoźywcze według jakości produkcji.
\\ \\
W uczeniu maszynowym klasyfikacja jest to metoda uczenia nadzorowanego, podczas której model próbuje przewidzieć etykietę obiektu na podstawie jego cech.\ Proces uczenia modelu klasyfikacji jest oparty o dwa zbiory, treningowy i testowy.\ Zbiór treningowy powinien być mniejszy od zbioru testowego.\ Po udanym procesie trenowania modelu następuje proces testowania, na podstawie którego wylicza się odpowiednie metryki pozwalające na ewaluacje modelu.\ W pracy magisterskiej wykorzystano zbiór danych, który posiada dwa typy etykiet [0, 1], dlatego też na tej podstawie zbudowano macierz pomyłek.

\section{Metryki klasyfikacji danych}
W trakcie ewaluacji algorytmu służącego do klasyfikacji wykorzystuje się metryki klasyfikacji.\ Służą one do przewidywanie etykiet klas na podstawie danych wejściowych.\ W zależności od ilości klas wyjściowych rozróżniamy klasyfikację binarną i wieloklasową.\ W klasyfikacji binarnej występują tylko dwie klasy wyjściowe, a w klasyfikacji wieloklasowej już tych klas jest więcej.\ Przykładem klasyfikacji binarnej jest np. wykrywanie niechcianych wiadomości w skrzynce mailowej.\ Wiadomość przychodzącą możemy w tedy określić jako 1 - \textit{,,pozytywną''}, bądź 0 - \textit{,,negatywną''}.\ Istnieje wiele sposobów na oceny klasyfikacji, do których zaliczamy m.in. dokładność, macierz pomyłek czy precyzję~\cite{Agrawal2024}.
\\ \\
Macierz pomyłek pokazuje jakie wyniki możemy otrzymać w procesie klasyfikacji.\ Macierz ta została przedstawiona w \refsource{tabeli}{tab:matrix-tn}.\ Jest to miara wydajności problemów klasyfikacji.

\begin{table}[H]
    \centering
    \captionsource{Macierz pomyłek}{Opracowanie własne}
    \label{tab:matrix-tn}
    \begin{tabular}{|c|c|M{16mm}|M{16mm}|}
        \hline
         & & \multicolumn{2}{c|}{\textbf{Prawdziwe wartości}} \\ \hline
         & & \textbf{1} & \textbf{0} \\ \hline
        \rule{0pt}{13mm} \multirow{2}{*}{\rotatebox[origin=c]{90}{\parbox{18mm}{\textbf{Przewidziane wartości}}}} & \textbf{1} & \cellcolor{lightgreen}TP & \cellcolor{lightred}FN \\ \cline{2-4}
        \rule{0pt}{13mm} & \textbf{0} & \cellcolor{lightred}FP & \cellcolor{lightgreen}TN \\ \hline
    \end{tabular}
\end{table}
    Wyróżniamy 4 rodzaje rezultatów:
    \begin{itemize}
        \item  \textbf{TP} - prawdziwie pozytywny - przewidywano pozytywny wynik i otrzymano pozytywny wynik,
        \item \textbf{FN} - fałszywie negatywny - przewidziano negatywny wynik, a otrzymano pozytywny wynik,
        \item \textbf{FP} - fałszywie pozytywny - przewidziano pozytywny wynik, a otrzymano negatywny wynik,
        \item \textbf{TN} - prawdziwie negatywny - przewidziano negatywny wynik i otrzymano negatywny wynik.
    \end{itemize}

Powyższe wartości pozwalają ocenić jakość algorytmów klasyfikujących dane.

\subsection{Dokładność}
\begin{equation}\label{math:acc}
    \frac{TP + TN}{TP + TN + FP + FN}
\end{equation}
Stosunek wszystkich dobrze oznaczonych obiektów do liczby wszystkich prób.

\subsection{Precyzja}
\begin{equation}\label{math:prec}
    \frac{TP}{TP + FP}
\end{equation}
Stosunek poprawnie wybranych obiektów klasy ,,\textit{1}'', do wszystkich wybranych obiektów tej klasy.

\subsection{Czułość}
\begin{equation}\label{math:rec}
    \frac{TP}{TP + FN}
\end{equation}
Stosunek poprawnie sklasyfikowanych obiektów klasy ,,\textit{1}'', do wszystkich obiektów, które powinny być w tej klasie.

\subsection{F1}
\begin{equation}\label{math:f1}
   \frac{2*x*y}{x + y}
\end{equation}
Jest średnia harmoniczna precyzji (\textit{x}) i czułości (\textit{y});


\subsection{AUC}
AUC - (\textit{Area Under Roc Curve}) - Jest to pole pod krzywą ROC, pokazuje sprawność klasyfikatora.\ Im wyższa wartość AUC tym lepiej.\ Wynik AUC jest z zakresu <0, 1>:
\begin{itemize}
    \item AUC = 1 - klasyfikator idealny,
    \item AUC = 0,5 - klasyfikator losowy,
    \item AUC < 0,5 - klasyfikator gorszy niż losowy~\cite{Algolytics}.
\end{itemize}