\pdfbookmark[0]{Streszczenie}{streszczenie.1}
\begin{abstract}
W ramach pracy magisterskiej utworzono środowisko badawcze potrzebne do przeprowadzenia analizy porównawczej algorytmów klasyfikacyjnych. Następnie przeprowadzono eksperyment, w którym dokonano klasyfikacji danych za pomocą wybranych algorytmów oraz zebrano pomiary jakości poszczególnych algorytmów. Na koniec przeanalizowano wyniki i wyciągnięto wnioski. Doświadczenie wykonano na platformie Azure Machine Learning Studio. Dane wykorzystane w badaniu pochodzą ze zbioru przygotowanego przez Kanadyjski Instytut Cyberbezpieczeństwa działającego przy Uniwersytecie Nowy Brunszwik. Powyższe działania zostały wykonane w celu sprawdzenia zasadności tworzenia jedynie autorskich rozwiązań w celu rozwiązania konkretnego problemu. Praca magisterska obejmuje również podsumowanie pracy inżynierskiej pt. ,,Wykorzystanie algorytmów genetycznych w systemach wykrywania intruzów w sieciach komputerowych'', przegląd środowisk chmurowych, low-code/no-code oraz algorytmów biorących udział w doświadczeniu.
\end{abstract}
\mykeywords

{
\selectlanguage{english}
\begin{abstract}
In the master's thesis was created a research environment necessary to conduct a comparative analysis of classification algorithms. Then an experiment was carried out in which data was classified using selected algorithms and quality measurements of individual algorithms were collected. Finally, the results were analyzed and conclusions were drawn. The experiment was performed on the Azure Machine Learning Studio platform. The data used in the study comes from the set prepared by Canadian Cybersecurity Institute operating at the University of New Brunswick. The above activities were carried out in order to check the validity of creating only proprietary solutions to solve a specific problem. The master's thesis also includes a summary of the engineering thesis entitled: ,,The use of genetic algorithms in intruder detection systems in computer networks'', a review of cloud, low-code/no-code environments and the algorithms involved in the experiment.
\end{abstract}
\mykeywords
}
