\documentclass[12pt]{pwrdpl}

\author{Bartosz Błyszcz}
\shortauthor{B. Błyszcz}

\titlePL{Wykorzystanie algorytmów genetycznych w systemach wykrywania intruzów w sieciach komputerowych}

\shorttitlePL{Wykorzystanie algorytmów genetycznych w systemach wykrywania intruzów}

\thesistype{PRACA DYPLOMOWA\\\vspace*{1mm}MAGISTERSKA}

\supervisor{dr inż. Tomasz Babczyński}

\degreeprogramme{Informatyka Techniczna}

\date{2024}

\faculty{Wydział Informatyki i Telekomunikacji}

\begin{document}

	\titlepages
	\RedefinePlainStyle
	
	\setcounter{tocdepth}{2}
	\tableofcontents
	\clearpage
	%\section*{Wykaz skrótów}

\begin{table}[H]
    \centering
    \captionsource{Tabela skrótów}{opracowanie własne}
    \begin{tabularx}{\linewidth}{lXX}
        \textbf{GA} & \textit{Genetic Algorithm} & Algorytm Genetyczny \\ \hline
        \textbf{GP} & \textit{Genetic Programming} & Programowanie Genetyczne \\ \hline
        \textbf{GNB} & \textit{Gaussian Naive Bayes} & Naiwny Klasyfikator Bayesa wykorzystujący rozkład Gaussa \\ \hline
        \textbf{ANN} & \textit{Artificial Neural Network} & Sztuczna sieć neuronowa \\ \hline
        \textbf{CNN} & \textit{Convolutional Neural Network} & Konwolucyjna sieć neuronowa \\ \hline
        \textbf{ML} & \textit{Machine Learning} & Uczenie maszynowe \\ \hline
        \textbf{AI} & \textit{Artificial Intelligence} & Sztuczna Inteligencja \\ \hline
        \textbf{IDS} & \textit{Intrusion Detection System} & System Wykrywania Intruzów \\ \hline
        \textbf{SVM} & \textit{Support Vector Machine} & Maszyna Wektorów Nośnych \\ \hline
        \textbf{AUC} & \textit{Area Under Roc Curve} & Przestrzeń pod krzywą ROC \\ \hline
        \textbf{LCDPs} & \textit{Low-code Development Platforms} & Platforma Low-code \\ \hline
        \textbf{BI} & \textit{Business Intelligence} & Narzędzia biznesowe do przekształcania danych \\ \hline
        \textbf{CDN} & \textit{Content Delivery Network} & Sieć dostarczania zawartości \\ \hline
        \textbf{MLP} & \textit{Multilayer Perceptron network} & Sieć wielowarstwowa perceptronowa \\ \hline
        \textbf{Azure ML} & \textit{Azure Machine Learning Studio} & \\
    \end{tabularx}
    \label{tab:shorts}
\end{table}
	
	\sloppy
	\setnowidow
	
	%\chapter{Wstęp}

\section{Wprowadzenie i uzasadnienie tematu pracy}
Klasyfikacja danych tabelarycznych jest zagadnieniem, które dostarcza wyzwań jej twórcom z powodu mnogości danych, a także mnogości cech, a także z nierzadko małą ilością próbek.\ Jednym z problemów jest między innymi dobór odpowiedniego algorytmu do problemu.\ Dane tabelaryczne występują w każdej dziedzinie, przez co raz na jakiś czas proponowane są nowe rozwiązania i algorytmy mające rozwiązać problem klasyfikacji w sposób lepszy i wydajny.\ Część twórców próbuje podchodzić do tego w sposób innowacyjny, lecz nie zawsze to wychodzi z powodu, chociażby dostosowania algorytmu pod konkretną strukturę danych, co powoduje problemy z wykorzystaniem rozwiązania dla innych danych.
\\ \\
Obecnie jednymi z najpopularniejszych algorytmów do klasyfikacji danych są logiczna regresja \trans{ang. logistic regression}, drzewo decyzyjne \trans{ang. decision tree}, losowy las \trans{ang. random forest}, maszyna wektorów nośnych \trans{ang. support vector machine}, naiwny Bayes \trans{ang. Naive Bayes}.\ Dlatego też bardzo ważne jest porównanie wytworzonego wcześniej rozwiązania z grupą innych algorytmów, które próbują przetworzyć ten sam zestaw danych.
\\ \\
W dzisiejszych czasach próba taka jest bardzo uproszczona, chociażby przez takie platformy jak \textit{Machine Learning Studio}, które pozwalają na wykorzystanie mocy obliczeniowej sklasteryzowanych jednostek wirtualnych do wykonywania obliczeń na odpowiednich maszynach wirtualnych, a także do budowania skomplikowanych zautomatyzowanych procesów złożonych z wielu zadań \trans{ang. pipeline}.\ W związku, z czym możliwość wykorzystania platformy chmurowej pozwoli na zautomatyzowanie procesu porównawczego oraz oddelegowanie zadań od chmury obliczeniowej co pozwoli na uniezależnienie powodzenia doświadczenia od mocy obliczeniowej komputera lokalnego, a także na ukazanie całościowo procesu porównania algorytmów klasyfikacyjnych.

\section{Cel pracy dyplomowej}
Celem niniejszej pracy dyplomowej jest porównanie algorytmu klasyfikacji danych tabelarycznych wypracowanego w trakcie pisania pracy inżynierskiej, do algorytmów dostępnych w aplikacji \textit{Machine Learning Studio} znajdującej się na platformie \textit{Microsoft Azure}.

\section{Założenie techniczne}

Dane prezentowane w \refsource{tabeli}{tab:technical} określają podstawowe założenia techniczne przyjęte w trakcie wykonywania analizy porównawczej.\ Dane te dotyczą między innymi środowiska, w którym wykonane było doświadczenie.\ Dodatkowo uwzględniono zestaw danych oraz biblioteki użyte w trakcie tworzenia doświadczenia.

\begin{table}[H]
    \centering
    \captionsource{Założenia techniczne pracy dyplomowej}{Opracowanie własne}
    \label{tab:technical}
    \begin{tabular}{|l|l|}
        \hline
        \textbf{Środowisko uruchomieniowe} & Machine Learning Studio\cite{azureml} \\ \hline
        \textbf{Język programowania} & Python 3.x \\ \hline
        \multirow{3}*{\textbf{Wykorzystane biblioteki}} & scikit-learn~\cite{scikit-learn} \\
        \cline{2-2}
        & Numpy~\cite{Harris2019} \\
        \cline{2-2}
        & Pandas~\cite{pandas, McKinney2010} \\
        \hline
        \textbf{Wykorzystane dane} & CICDS2017~\cite{cicds2017kaggle} \\
        \hline
    \end{tabular}
\end{table}
	%\include{ids2}
	%\include{genetic2}
	%\include{klasyfikatory}
	%\include{badania}
	%\include{przebieg}
	%\chapter{Perspektywy rozwoju}
Stworzony projekt jest jedynie silnikiem klasyfikacyjnym, który pozwala na wytrenowanie i wyłonienie najlepszego algorytmu do klasyfikacji danych.\ Dzięki możliwościom platformy Azure ML stworzenie całego środowiska testowego jest relatywnie tanie i nie wymaga wielu wyspecjalizowanych umiejętności.\ Bazując na wynikach i najlepszym klasyfikatorze, można utworzyć odpowiednie przepływy służące do klasyfikacji danych wejściowych.\ Dostęp do nich może odbywać się za pomocą utworzonych Punktów Dostępowych \trans{ang. Endpoints}.\ Dzięki punktom dostępowym możliwa jest komunikacja za pomocą protokołu HTTPS \trans{ang. Hyper Text Transfer Protocol Secure} i komunikacja typu REST \textit{(ang. Representative State Transfer)}.
\\ \\
Utworzony w ten sposób punkt dostępu może zostać wykorzystany w klasyfikacji ruchu sieciowego w niewielkich aplikacjach z dostępem do internetu.\ Pozwoliłoby to na realizację analizy danych w chmurze, co mogłoby przyspieszyć cały proces oraz utworzyć pojedyncze źródło prawdy dla wielu instancji aplikacji.\ A wykorzystanie dodatkowo konteneryzacji, którą zapewnia platforma Azure, cały proces mógłby zostać zoptymalizowany pod kątem wydajnościowym i lokalizacyjnym.\ Pojedyncze źródło prawdy jest zaletą wykorzystania ,,\textit{zewnętrznego}'' klasyfikatora, ponieważ zapewnia jednakowe wyniki klasyfikacji w poszczególnych instancjach samej aplikacji instalowanej na wielu urządzeniach.\ Pozwala to również na lepsze dostrajanie całego procesu, a wykorzystanie kolejnych wersji przepływów umożliwia utrzymywanie kopii zapasowych poszczególnych rozwiązań.\ Umożliwia to na przykład cofnięcie wersji klasyfikatora w przypadku wykrycia nieprawidłowości w obecnym modelu.

	% \include{rozdzial3}
	% \include{tests}
	Jak to super jest to kompilować xd Jak to super jest to kompilować xd Jak to super jest to kompilować xd Jak to super jest to kompilować xd Jak to super jest to kompilować xd Jak to super jest to kompilować xd Jak to super jest to kompilować xd Jak to super jest to kompilować xd Jak to super jest to kompilować xd Jak to super jest to kompilować xd Jak to super jest to kompilować xd Jak to super jest to kompilować xd Jak to super jest to kompilować xd Jak to super jest to kompilować xd
	% itd.
	% \appendix
	% \include{dodatekB}
	% itd.
	\begin{table}[H]
		\centering
		\caption{Tabelka}
		\begin{tabular}{|l|r|} \hline
			\textbf{Label} & \textbf{Liczba} \\ \hline
			\textbf{accuracy} & $11,1\%$ \\ \hline
		\end{tabular}
	\end{table}


	\renewcommand*{\listfigurename}{Wykaz rysunków}
	\renewcommand*{\listtablename}{Wykaz tabel}
	\listoffigures
	\clearpage
	\listoftables

	\printbibliography

\end{document}
