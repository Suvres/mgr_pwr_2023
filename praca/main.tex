\documentclass[12pt]{pwrdpl}

\author{Bartosz Błyszcz}
\shortauthor{B. Błyszcz}

\titlePL{Wykorzystanie algorytmów genetycznych w systemach wykrywania intruzów w sieciach komputerowych}

\shorttitlePL{Wykorzystanie algorytmów genetycznych w systemach wykrywania intruzów}

\thesistype{PRACA DYPLOMOWA\\\vspace*{1mm}MAGISTERSKA}

\supervisor{dr inż. Tomasz Babczyński}

\degreeprogramme{Informatyka Techniczna}

\date{2024}

\faculty{Wydział Informatyki i Telekomunikacji}

\begin{document}

	\titlepages
	\RedefinePlainStyle
	
	\setcounter{tocdepth}{2}
	\tableofcontents
	\clearpage
	%\section*{Wykaz skrótów}

\begin{table}[H]
    \centering
    \begin{tabularx}{\linewidth}{lXX}
        \textbf{Skrót} & \textbf{Nazwa angielska} & \textbf{Nazwa polska} \\ \hline
        \textbf{GA} & \textit{Genetic Algorithm} & Algorytm Genetyczny \\ \hline
        \textbf{GP} & \textit{Genetic Programming} & Programowanie Genetyczne \\ \hline
        \textbf{IDS} & \textit{Intrusion Detection System} & System wykrywania intruzów  \\ \hline
        \textbf{IPS} & \textit{Intrusion Prevent System} & System zapobiegania włamaniom \\ \hline
        \textbf{GNB} & \textit{Gaussian Naive Bayes} & Naiwny Klasyfikator Bayesa wykorzystujący rozkład Gaussa \\ \hline
        \textbf{ANN} & \textit{Artificial Neural Network} & Sztuczna sieć neuronowa \\ \hline
        \textbf{CNN} & \textit{Convolutional Neural Network} & Konwolucyjna sieć neuronowa \\ \hline
        \textbf{ML} & \textit{Machine Learning} & Uczenie maszynowe \\ \hline
        \textbf{AI} & \textit{Artificial Intelligence} & Sztuczna Inteligencja \\ \hline
        \textbf{IDS} & \textit{Intrusion Detection System} & System Wykrywania Intruzów \\ \hline
        \textbf{SVM} & \textit{Support Vector Machine} & Maszyna Wektorów Nośnych \\ \hline
        \textbf{AUC} & \textit{Area Under Roc Curve} & Przestrzeń pod krzywą ROC \\ \hline
        \textbf{LCDPs} & \textit{Low-code Development Platforms} & Platforma Low-code \\ \hline
        \textbf{BI} & \textit{Business Intelligence} & Narzędzia biznesowe do~przekształcania danych \\ \hline
        \textbf{CDN} & \textit{Content Delivery Network} & Sieć dostarczania zawartości \\ \hline
        \textbf{MLP} & \textit{Multilayer Perceptron network} & Sieć wielowarstwowa perceptronowa \\ \hline
        \textbf{Azure ML} & \textit{Azure Machine Learning Studio} & Azure Machine Learning Studio \\ \hline
        \textbf{GAGNB} & \textit{Gaussian Naive Bayes - with GA} & Gaussian Naive Bayes - with GA\\ \hline
        \textbf{HTTPS} & \textit{Hyper Text Transfer Protocol Secure} & Protokół służący do komunikacji w sieci internetowej \\ \hline
        \textbf{REST} & \textit{Representative State Transfer} & Rozwiązanie architektoniczne służące do komunikacji w sieci internetowej \\ \hline
    \end{tabularx}
    \label{tab:shorts}
\end{table}
	
	\sloppy
	\setnowidow
	
	%\chapter{Wstęp}

\section{Wprowadzenie i uzasadnienie tematu pracy}
Klasyfikacja danych tabelarycznych jest trudnym zagadnieniem do analizy, z powodu powszechnego występowania bardzo dużej ilości nieuporządkowanych danych, które zawierają wiele cech.\ To proces organizowania danych w tabeli w celu ich łatwiejszej analizy, interpretacji czy dalszego przetwarzania.\ Podczas takiej kategoryzacji danych ważne jest właściwe dobranie algorytmu, ze względu na typ danych.\ Przykładowo zbiór danych tekstowych można klasyfikować za pomocą jednokierunkowej sieci neuronowej, a obrazy za pomocą sieci konwolucyjnej.\
\\ \\
Obecnie istnieje wiele różnych algorytmów do klasyfikacji danych tabelarycznych.\ Jednymi z popularniejszych są: regresja logistyczna \trans{ang. logistic regression}, drzewo decyzyjne \trans{ang. decision tree}, las losowy \trans{ang. random forest}, maszyna wektorów nośnych \trans{ang. support vector machine}, naiwny klasyfikator Bayesowski \trans{ang. Naive Bayes classifier}.\ Przy wykorzystaniu tych algorytmów do kategoryzacji, ważne jest właściwe wybranie algorytmu, czyli rozpoznanie z jakimi danymi mamy do czynienia oraz porównanie wyników klasyfikacji, w celu wybrania najlepszego dopasowania.
\\ \\
Dane tabelaryczne występują w każdej dziedzinie, duże zestawy danych można spotkać w medycynie, nauce czy w finansach.\ Rosnąca liczba danych oraz ich zmienna struktura wymaga opracowywania coraz lepszych algorytmów klasyfikacji.\ Jednakże wyjątkowość danych sprawia, że trudno opracować uniwersalny algorytm klasyfikacji.\ Wiele rozwiązań jest tworzonych dla konkretnej struktury danych, co powoduje niemożność ich wykorzystania dla innych danych.
\\ \\
Przy rosnącej liczbie danych do analizy, rozwijają się metody ułatwiające ich klasyfikację. Coraz częściej wykorzystuje się rozwiązania z zakresu sztucznej inteligencji czy obliczeń chmurowych. Jednym z przykładów jest aplikacja \textit{Machine Learning Studio},  dostarczana przez \textit{Microsoft Azure}. Zawiera ona zestaw narzędzi, umożliwiających łatwiejsze kategoryzowanie danych czy tworzenie algorytmów klasyfikacji i ich porównywanie. Użycie chmury pozwala na wykorzystanie mocy obliczeniowej sklasteryzowanych jednostek wirtualnych do wykonywania obliczeń na odpowiednich maszynach wirtualnych czy do budowania skomplikowanych zautomatyzowanych procesów złożonych z wielu zadań \trans{ang. pipeline}. Natomiast wykorzystanie sztucznej inteligencji pozwala na wprowadzenie elementu uczenia się w celu lepszego rozpoznawania danych.
\\ \\
Zastosowanie tych narzędzi umożliwia automatyzację procesu badawczego, porównanie wyników działania różnych algorytmów oraz znaczne przyspieszenie badań.\ Ma to znaczenie przy rosnącym zaprotrzebowaniu na analizę dużych zestawów danych.
\\ \\
W dobie rozwijających się hurtowni danych oraz sztucznej inteligencji coraz więcej firm przetrzymuje ogromne zbiory danych w sieci komputerowej.\ Dane takie mogą być poufne bądź o znaczeniu strategicznym.\ Wyciek takich danych może mieć negatywne konsekwencje dla właścicieli danych bądź osób, których dane sa przechowywane.\ Przykładem może być dostęp do prywatnych kont bankowych albo danych medycznych.\ Dlatego też dane te są zabezpieczane m.in. przy użyciu kryptografii.\ Pozwala to na zaszyfrowanie przesyłu danych poufnych jak i samych przesyłanych danych.\ Jednakże osobom planującym kradzież konkretnych danych dużo bardziej zależy na uzyskanie nieautoryzowanego dostępu do komputerów mogących posiadać dostęp do danych wrażliwych.\ Dzieje się tak, ponieważ dostęp do urządzeń, pozwala na wgląd nie tylko do konkretnych danych ale i do całej sieci intranetowej np. banku.\ Dostęp taki może zostać również wykorzystany do np. wgrania szkodliwego oprogramowania umożliwiającej założenie ,,\textit{tylnej furtki}'' \textit{(ang. backdoor)}.\ W celu ograniczenia możliwych nieautoryzowanych dostępów z zewnątrz powstało oprogramowanie \textit{IDS, IPS} \textit{(ang. Intrusion Detection system, Intrusion Prevent System)}.\ Systemy do ostrzegania i przeciwdziałania atakom na sieć komputerową~\cite{Blyszcz2022}.
\\ \\
Dane sieciowe są bardzo złożone oraz posiadają dużo cech, dlatego do ich analizy można wykorzystać algorytmy uczenia maszynowego pozwalające na wykrycie nietypowego ruchu sieciowego.\ Klasyczne systemy IDS bazują głównie na regułach wykluczających konkretny ruch sieciowy.\ Przewagą uczenia maszynowego jest to, że może wykrywać anomalie niewchodzące w skład reguł bezpieczeństwa w sieci.\ Może to umożliwić przy niewielkim koszcie wytrenowania sieci wczesne i szybsze wykrywanie potencjalnych ataków niż w przypadku klasycznyc hsystemów IDS opartych o reguły.

\section{Cel i zakres pracy}
Celem niniejszej pracy dyplomowej jest ocena jakości opracowanego w pracy inżynierskiej autorskiego sposobu klasyfikacji danych tabelarycznych, wykorzystującego algorytm genetyczny oraz Klasyfikator Naiwny Bayesa.\ W tym celu dokonano analizy porównawczej rozwiązania wraz z algorytmami dostępnymi w aplikacji \textit{Machine Learning Studio}. Algorytmy opisano w Podrozdziale~\ref{sec:alg}.
\\ \\
Praca składa się z 2 części.\ W części teoretycznej (Rozdziały 2 - 5) dokonano przeglądu dostępnych rozwiązań chmurowych (podejścia low-code/no-code).\ Oprócz tego opisano zagadnienia związane ze sztuczną inteligencją.\ W części badawczej (Rozdziały 6 - 8) przygotowano programistyczne stanowisko badawcze.\ W tym celu scharakteryzowano metryki jakościowe i opracowano eksperyment.\ Wykonano analizę porównawczą i statystyczną otrzymanych wyników.\ Dane wykorzystane do badań pochodziły z Instytutu Cyberbezpieczeństwa, działającego przy Uniwersytecie Nowy Brunszwik.


	%\include{ids2}
	%\include{genetic2}
	%\include{klasyfikatory}
	%\include{badania}
	%\include{przebieg}
	%\chapter{Podsumowanie}
Niniejsza praca magisterska miała na celu, wykonanie porównania algorytmów klasyfikacyjnych dostarczonych przez platformę Azure, wraz z GAGNB oraz algorytmem DANet.\ Wśród autorskich algorytmów znajdywał się algorytm opracowany przez autora pracy magisterskiej - \textit{Gaussian Naive Bayes - with GA}.\ Miało to na celu sprawdzenie, czy tworzenie autorskich rozwiązań nakierowanych na problem będzie opłacalne w dobie gotowych rozwiązań.
\\ \\
W pracy dokonano przeglądu literaturowego związanego zagadnieniem uczenia maszynowego oraz z podejściem low-code/no-code.\ Dzięki temu wybrano narzędzie Machine Learning Studio znajdujące się na platformie Microsoft Azure. \ Microsoft wyszedł naprzeciw potrzebom użytkowników, przygotowując zestaw prekonfigurowanych algorytmów klasyfikacyjnych.\ Możliwości narzędzia Azure ML pozwalają na tworzenie wysokoskalowalnych rozwiązań z zakresu uczenia maszynowego przy relatywnie niskich kosztach.\ Jednakże mogą wystąpić specyficzne wymagania biznesowe albo prawne, które nie będą pozwalały na zastosowanie zewnętrznych narzędzi chmurowych.\ Takim przykładem są strategiczne dane państwowe, których wyciek za granicą może grozić poważnym zagrożeniem z zewnątrz.\ W przypadku systemów wykrywania intruzów zasadne jest stosowanie autorskich rozwiązań, które nie są znane opinii publicznej.\ Takie działanie chroni przed nieautoryzowanym dostępem do sieci komputerowej oraz do danych wrażliwych.
\\ \\
Konkurencyjność rozwiązań autorskich przedstawiono na \refsource{wykresie}{fig:predict-result}.\ Wykorzystanie połączenia algorytmu genetycznego i klasyfikatora naiwnego Bayesa z rozkładem normalnym pozwala na uzyskanie zbliżonych wyników do algorytmu utworzonych przez Microsoft.\ Różnica około 5 punktów procentowych między najlepszym algorytmem a algorytmem \textit{Gaussian Naive Bayes - with GA} ukazuje niewielką różnicę w jakości algorytmu.\ Częstą wadą autorskich rozwiązań jest ich nakierowanie na konkretny zestaw danych.\ Zostało to pokazane podczas wykorzystania algorytmu DANet, który dobrze sklasyfikował dane związane z m.in. chorobami serca~\cite{Chen2022}.\ Algorytm ten jednakże błędnie skategoryzował dane związane z ruchem sieciowym.
\\ \\
Dodatkowo korzystanie z tego typu prostych rozwiązań autorskich pozwala na prototypownie rozwiązań biznesowych opartych o klasyfikację danych.\ Zastosowanie algorytmu \textit{GAGNB} nie wymaga wcześniejszej znajomości zbioru danych.\ Pozwala na lokalne korzystanie z programu do klasyfikacji, bez ponoszenia kosztów wykorzystania platformy chmurowej.\ Kolejnym atutem tego algorytmu jest zmniejszenie kosztów lokalnego użytkowania.\ Co zostało spowodowane zmniejszeniem wymiarowości zbioru danych do klasyfikacji poprzez wykorzystanie jedynie wytypowanych kolumn.
\\ \\
Stworzony projekt jest jedynie silnikiem klasyfikacyjnym, który pozwala na wytrenowanie i wyłonienie najlepszego algorytmu do klasyfikacji danych.\ Dzięki możliwościom platformy Azure ML stworzenie całego środowiska testowego jest relatywnie tanie i nie wymaga wielu wyspecjalizowanych umiejętności.\ Bazując na wynikach i najlepszym klasyfikatorze, można utworzyć odpowiednie przepływy służące do klasyfikacji danych wejściowych.\ Dostęp do nich może odbywać się za pomocą utworzonych Punktów Dostępowych \trans{ang. Endpoints}.\ Dzięki punktom dostępowym możliwa jest komunikacja za pomocą protokołu HTTPS \trans{ang. Hyper Text Transfer Protocol Secure} i komunikacja typu REST \textit{(ang. Representative State Transfer)}.
\\ \\
Utworzony w ten sposób punkt dostępu może zostać wykorzystany w klasyfikacji ruchu sieciowego w niewielkich aplikacjach z dostępem do internetu.\ Pozwoliłoby to na realizację analizy danych w chmurze, co mogłoby przyspieszyć cały proces oraz utworzyć pojedyncze źródło prawdy dla wielu instancji aplikacji.\ A wykorzystanie dodatkowo konteneryzacji, którą zapewnia platforma Azure, cały proces mógłby zostać zoptymalizowany pod kątem wydajnościowym i lokalizacyjnym.\ Pojedyncze źródło prawdy jest zaletą wykorzystania ,,\textit{zewnętrznego}'' klasyfikatora, ponieważ zapewnia jednakowe wyniki klasyfikacji w poszczególnych instancjach samej aplikacji instalowanej na wielu urządzeniach.\ Pozwala to również na lepsze dostrajanie całego procesu, a wykorzystanie kolejnych wersji przepływów umożliwia utrzymywanie kopii zapasowych poszczególnych rozwiązań.\ Umożliwia to na przykład cofnięcie wersji klasyfikatora w przypadku wykrycia nieprawidłowości w obecnym modelu.

	% \include{rozdzial3}
	% \include{tests}
	Jak to super jest to kompilować xd Jak to super jest to kompilować xd Jak to super jest to kompilować xd Jak to super jest to kompilować xd Jak to super jest to kompilować xd Jak to super jest to kompilować xd Jak to super jest to kompilować xd Jak to super jest to kompilować xd Jak to super jest to kompilować xd Jak to super jest to kompilować xd Jak to super jest to kompilować xd Jak to super jest to kompilować xd Jak to super jest to kompilować xd Jak to super jest to kompilować xd
	% itd.
	% \appendix
	% \include{dodatekB}
	% itd.
	\begin{table}[H]
		\centering
		\caption{Tabelka}
		\begin{tabular}{|l|r|} \hline
			\textbf{Label} & \textbf{Liczba} \\ \hline
			\textbf{accuracy} & $11,1\%$ \\ \hline
		\end{tabular}
	\end{table}


	\renewcommand*{\listfigurename}{Wykaz rysunków}
	\renewcommand*{\listtablename}{Wykaz tabel}
	\listoffigures
	\clearpage
	\listoftables

	\printbibliography

\end{document}
