\chapter{Streszczenie pracy inżynierskiej}
W ramach pracy inżynierskiej opracowano autorski algorytm \textit{Gaussian Naive Bayes - with GA} (\textit{GAGNB}), wykorzystujący algorytm genetyczny oraz algorytm Gaussian Naive Bayes (\textit{GNB}). Do badań wykorzystano dane uzyskane z Instytutu Cyberbezpieczeństwa, działającego przy Uniwersytecie Nowy Brunszwik. Dane użyto też w niniejszej pracy dyplomowej. Zbiór danych został opisany w \refsource{sekcji}{sec:data}.

\section{Algorytm genetyczny}
Algorytm genetyczny jest metodą optymalizacji. Może być zastosowany w procesie klasyfikacji danych. Algorytm ten przeszukuje przestrzeń alternatywnych rozwiązań dla danego problemu, by znaleźć najlepszy
wynik~\cite{Kusiak2021}. Działanie algorytmu genetycznego jest inspirowane między innymi ewolucją biologiczną, stąd też pochodzą nazwy kolejnych kroków, które zostały przedstawione na \refsource{rysunku}{fig:GA}.


\begin{figure}[H]
    \centering
    \resizebox{0.6\textwidth}{!}{
    \begin{tikzpicture}[node distance={2cm}]
        \node (start) [startstop] {Start};
        \node (init) [process, below of=start] {Inicjalizacja};
        \node (eval) [process, below of=init] {Ewaluacja};
        \node (opt) [decision, below of=eval, aspect=3, yshift=-5mm] {Czy wynik\\jest optymalny};
        \node (select) [process, below of=opt, yshift=-10mm] {Selekcja};
        \node (rec) [process, below of=select] {Rekombinacja};
        \node (mut) [process, below of=rec] {Mutacja};
        \node (rep) [process, below of=mut] {Zamiana};
        \node (end) [startstop, below left of=opt, xshift=-4.5cm] {Koniec};
        \draw [arrow] (start) -- (init);
        \draw [arrow] (init) -- (eval);
        \draw [arrow] (eval) -- (opt);
        \draw [backarrow] (end) |- node[anchor=south, darkgreen, xshift=2mm] {tak} (opt);
        \draw [arrow] (opt) -- node[anchor=east, red] {nie} (select);
        \draw [arrow] (select) -- (rec);
        \draw [arrow] (rec) -- (mut);
        \draw [arrow] (mut) -- (rep);
        \draw [arrow] (rep.east) -| ++(3, 11.5) -- (eval.east);
    \end{tikzpicture}}
    \captionsource{Schemat algorytmu genetycznego}{Opracowanie własne na podstawie:~\cite{Sastry2005,Kusiak2021, Blyszcz2022}}
    \label{fig:GA}
\end{figure}
Poniżej opisano przebieg algorytmu genetycznego. Proces składa się z 6 etapów:
\begin{enumerate}
    \item \textbf{Inicjalizacja} -- metoda inicjująca populację losowych osobników, wykorzystywanych podczas obliczeń;
    \item \textbf{Ewaluacja} -- etap sprawdzenia jakości wygenerowanych danych;
    \item \textbf{Selekcja} -- część algorytmu skupiająca się na wybraniu $n$ najlepszych rozwiązań, które przejdą do następnej populacji;
    \item \textbf{Rekombinacja} -- element tworzący nową populację na bazie poprzedniej populacji. Wykorzystuje do tego mechanizm krzyżowania rodziców;
    \item \textbf{Mutacja} -- metoda wprowadzająca zmianę w $n$ losowych miejsc w genomie;
    \item \textbf{Zamiana} -- miejsce wymiany starej populacji na nową~\cite{Sastry2005, Kusiak2021, Blyszcz2022}.
\end{enumerate}


\section{Gaussian Naive Bayes}
GNB to implementacja klasyfikatora naiwnego Bayesa opartego o rozkład normalny Gaussa.
Twierdzenie Bayesa zostało przedstawione na \refsource{równaniu}{math:bayes}.
\vfill
\pagebreak

\begin{equation}\label{math:bayes}
P(H|X) = \frac{P(X|H) P(H)}{P(X)}
\end{equation}
gdzie:
\begin{itemize}
    \item $H$ oraz $X$ są różnymi zdarzeniami;
    \item $P(X) \neq 0$;
    \item $P(H|X)$ -- prawdopodobieństwo wystąpienia zdarzenia $H$ jeśli zdarzenie $X$ wystąpiło;
    \item $P(X|H)$ -- prawdopodobieństwo wystąpienia zdarzenia $X$ jeśli zdarzenie $H$ wystąpiło;
    \item $P(H)$ oraz $P(X)$ to prawdopodobieństwa zaobserwowane, bez żadnych warunków~\cite{Leung2007}.
\end{itemize}
\ \\
Algorytm GNB wykorzystuje funkcję gęstości prawdopodobieństwa do wyliczenia szansy na przynależność od odpowiedniej klasy. Funkcja została opisana za pomocą \refsource{równania}{math:gnb}~\cite{Joyce2003}.
\begin{equation}\label{math:gnb}
    P(x_{i}|y) = \frac{1}{\sqrt{2\pi\sigma_{y}}}e^{-\frac{(x_{i}-\mu_{y})^2}{2\sigma_{y}^{2}}}
\end{equation}
gdzie:
\begin{itemize}
        \item[] \textbf{$x_{i}$} -- wartość cechy
        \item[] \textbf{$y$} -- reprezentacja klasy dla danej cechy
        \item[] \textbf{$\sigma_{y}^{2}$} -- wariancja wartości cechy dla danej klasy,
        \item[] \textbf{$\sigma_{y}$} -- odchylenie standardowe wartości cechy dla danej klasy,
        \item[] \textbf{$\mu_{y}$} -- średnia wartość cechy dla danej klasy~\cite{Leung2007}.
\end{itemize}
\ \\
Algorytm \textit{Gaussian Naive Bayes} wykorzystuje funkcję gęstości prawdopodobieństwa do obliczania $P(x_i)$ dla każdej klasy. Algorytm po wyliczeniu wszystkich składowych wymnaża poszczególne wartości dla danej klasy (\refsource{równanie}{math:gnb2}).

\begin{equation}\label{math:gnb2}
P(x_1,x_2,x_3,...,x_n|A) = \prod_{i=1}^{n}P(x_i|A)
\end{equation}
\ \\
Jednakże z powodu częstego mnożenia bardzo małych liczb może powodować błędy z zaokrąglaniem podczas obliczeń komputerowych, dlatego stosuje się sumę logarytmów poszczególnych wartości (\refsource{równanie}{math:gnb3}). Klasa z najwyższym wynikiem prawdopodobieństwa jest klasą wynikową~\cite{GNBalg}.

\begin{equation}\label{math:gnb3}
log(P(x_1,x_2,x_3,...,x_n|A)) = \sum_{i=1}^{n}log(P(x_i|A))
\end{equation}

\section{Autorski algorytm opracowany w ramach pracy inżynierskiej}
Algorytm opracowany w ramach pracy inżynierskiej wykorzystuje w procesie optymalizacji algorytm genetyczny oraz GNB. Celem algorytmu była redukcja wymiarowości danych tabelarycznych, by zwiększyć jakość klasyfikacji danych. Uzyskano to poprzez wyznaczenie najistotniejszych cech zbioru. Algorytm genetyczny zastosowano w celu przygotowania zbioru prawdopodobnie istotnych cech poprzez oznaczenie ich pozycji za pomocą ciągu cyfr 0 i 1. Cechy oznaczone cyfrą 1 były brane pod uwagę w procesie ewaluacji za pomocą GNB. Cały proces trwał maksymalnie 1000 iteracji lub do momentu uzyskania minimum $90\%$ dopasowania. Wynikiem działania autorskiego algorytmu jest zbiór cech istotnych w procesie klasyfikowania.

\section{Systemy wykrywania intruzów}
Systemy wykrywania intruzów \textit{(ang. Intrusion Detection System, IDS)} powstały w celu monitorowania sieci komputerowych. Ich zadaniem jest, rozpoznawanie szkodliwego ruchu sieciowego, detekcja zagrożeń oraz reagowanie na występujące anomalie. Klasyczny model wykrywania intruzów zawiera:
\begin{itemize}
\item \textbf{źródła informacji} służące do określenia czy doszło do ataku,
\item \textbf{moduł analizujący} przetwarzający zebrane dane,
\item \textbf{moduł odpowiadający} reagujący w określony sposób na wykryte anomalie~\cite{SazzadulHoque2012, Bacer, Blyszcz2022}.
\end{itemize}
IDS dzielimy na dwie grupy rozróżnione na podstawie sposobu działania. Są to:
\begin{itemize}
    \item \textbf{systemy oparte na hoście} \textit{(ang. Host Base Intrusion Detection System, HIDS)} - służą do zbierania informacji o zdarzeniach systemowych i wykrywania luk w systemach komputerowych,
    \item \textbf{systemy oparte na sieci komputerowej} \textit{(ang. Network Base Intrusion Detection System, NIST)} - służą do wykrywania szkodliwej aktywności w sieci~\cite{Blyszcz2022, SazzadulHoque2012, chawlaashima}.
\end{itemize}


\section{Omówienie  wyników uzyskanych w pracy inżynierskiej}
W celu sprawdzenia jakości własnego rozwiązania wykorzystano następujące metody statystyczne:

\begin{itemize}
    \item \textbf{metoda ANOVA} - analiza wariancji,
    \item \textbf{współczynnik korelacji Pearsona} - współczynnik określający zależność liniową pomiędzy zmiennymi losowymi,
    \item \textbf{współczynnik korelacji rang Spearmana} - współczynnik korelacji rang Pearsona, dla danych po ustaleniu rang.
\end{itemize}

Analiza wykazała, że wyniki uzyskane za pomocą metod statystycznych były gorsze niż autorskiego algorytmu. Rezultaty badań przedstawia \refsource{tabela}{tab:monday-workingHours} oraz \refsource{wykres}{fig:mond}.\

\begin{table}[H]
    \centering
    \captionsource{Klasyfikacja zbioru danych: Monday-WorkingHours}{\cite{Blyszcz2022}}
    \resizebox{\textwidth}{!}{
    \begin{tabular}{|l|r|r|r|r|r|r|r|}
        \hline
        \multicolumn{8}{|c|}{\textbf{Klasyfikacja: Monday-WorkingHours}} \\ \hline
        \multirow{2}*{} & \multirow{2}*{\textbf{Podstawowa}} & \multicolumn{3}{c|}{\textbf{Zoptymalizowana GA}} & \multicolumn{3}{c|}{\textbf{Zoptymalizowana Statystycznie}} \\
        \cline{3-8}
        & & \textbf{1} & \textbf{2} & \textbf{3} & \textbf{ANOVA} & \textbf{PEARSON} & \textbf{SPEARMAN} \\ \hline
        \textbf{Rozmiar danych [MB]} & 347,19 & 197.60 & 181,43 & 189,51 & \multicolumn{3}{c|}{197,60}\\ \hline
        \textbf{Ilość linii [-]} & \multicolumn{7}{c|}{529 918} \\ \hline
        \textbf{Czas operacji [s]} & 2,68 & 1,84 & 1,77 & 2,23 & 1,89 & 1,83 & 1,91 \\ \hline
        \textbf{Dokładność [\%]} & 45,89 & 81,71 & 85,47 & 88,67 & 39,32 & 36,34 & 37,77 \\ \hline
        \textbf{Precyzja [\%]} & \multicolumn{7}{c|}{100} \\ \hline
        \textbf{Czułość [\%]} & 45,89 & 81,71 & 85,47 & 88,67 & 39,32 & 36,34 & 37,77 \\ \hline
        \textbf{F1 [\%]} & 62,91 & 89,94 & 92,16 & 93,99 & 56,45 & 53,31 & 54,83 \\ \hline
        \textbf{Zużycie pamięci [MB]} & 1853,44 & 980,06 & 885,42 & 932,74 & \multicolumn{3}{c|}{980,06} \\ \hline
    \end{tabular}}
    \label{tab:monday-workingHours}
\end{table}

\refsource{Wykres}{fig:mond}. przedstawia wyniki dokładności. Zastosowano następujące opisy:
\begin{itemize}
    \item \textbf{GNB} - dopasowanie danych w podstawowym zbiorze;
    \item \textbf{GA\_GNB 1} - dopasowanie danych w pierwszej próbie wykorzystania algorytmu autora;
    \item \textbf{GA\_GNB 2} - dopasowanie danych w drugiej próbie wykorzystania algorytmu autora;
    \item \textbf{GA\_GNB 3} - dopasowanie danych w trzeciej próbie wykorzystania algorytmu autora;
    \item \textbf{ANOVA} - dopasowanie danych z wykorzystaniem cech uzyskanych metodą ANOVA;
    \item \textbf{PEARSON} - dopasowanie danych z wykorzystaniem cech uzyskanych metodą współczynników korelacji Pearsona;
    \item \textbf{SPEARMAN} - dopasowanie danych w zbiorze zmodyfikowanym o rozwiązanie uzyskane metodą  współczynników korelacji rang Spearmana.
\end{itemize}


\begin{figure}[H]
    \centering
    \includegraphics[width=0.7\textwidth]{images/Monday-WorkingHours_cmp}
    \captionsource{Klasyfikacja zbioru danych: Monday-WorkingHours}{\cite{Blyszcz2022}}
    \label{fig:mond}
\end{figure}

Analizując powyższe wyniki można zauważyć, że najwyższe dopasowanie uzyskał autorski algorytm a najniższe - zbiór cech wyznaczonych metodą współczynników korelacji Pearsona. Świadczy to o wysokiej jakości opracowanego w pracy inżynierskiej algorytmu.

