\chapter{Perspektywy rozwoju}
Stworzony projekt jest jedynie silnikiem klasyfikacyjnym, który pozwala na wytrenowanie i wyłonienie najlepszego algorytmu do klasyfikacji danych.\ Dzięki możliwościom platformy Azure ML stworzenie całego środowiska testowego jest relatywnie tanie i nie wymaga wielu wyspecjalizowanych umiejętności.\ Bazując na wynikach i najlepszym klasyfikatorze, można utworzyć odpowiednie przepływy służące do klasyfikacji danych wejściowych.\ Dostęp do nich może odbywać się za pomocą utworzonych Punktów Dostępowych \trans{ang. Endpoints}.\ Dzięki punktom dostępowym możliwa jest komunikacja za pomocą protokołu HTTPS \trans{ang. Hyper Text Transfer Protocol Secure} i komunikacja typu REST \textit{(ang. Representative State Transfer)}.
\\ \\
Utworzony w ten sposób punkt dostępu może zostać wykorzystany w klasyfikacji ruchu sieciowego w niewielkich aplikacjach z dostępem do internetu.\ Pozwoliłoby to na realizację analizy danych w chmurze, co mogłoby przyspieszyć cały proces oraz utworzyć pojedyncze źródło prawdy dla wielu instancji aplikacji.\ A wykorzystanie dodatkowo konteneryzacji, którą zapewnia platforma Azure, cały proces mógłby zostać zoptymalizowany pod kątem wydajnościowym i lokalizacyjnym.\ Pojedyncze źródło prawdy jest zaletą wykorzystania ,,\textit{zewnętrznego}'' klasyfikatora, ponieważ zapewnia jednakowe wyniki klasyfikacji w poszczególnych instancjach samej aplikacji instalowanej na wielu urządzeniach.\ Pozwala to również na lepsze dostrajanie całego procesu, a wykorzystanie kolejnych wersji przepływów umożliwia utrzymywanie kopii zapasowych poszczególnych rozwiązań.\ Umożliwia to na przykład cofnięcie wersji klasyfikatora w przypadku wykrycia nieprawidłowości w obecnym modelu.
