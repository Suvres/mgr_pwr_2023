\chapter{Podsumowanie}
Niniejsza praca magisterska miała na celu, wykonanie porównania algorytmów klasyfikacyjnych dostarczonych przez platformę Azure, wraz z GAGNB oraz algorytmem DANet.\ Wśród autorskich algorytmów znajdywał się algorytm opracowany przez autora pracy magisterskiej - \textit{Gaussian Naive Bayes - with GA}.\ Miało to na celu sprawdzenie, czy tworzenie autorskich rozwiązań nakierowanych na problem będzie opłacalne w dobie gotowych rozwiązań.
\\ \\
W pracy dokonano przeglądu literaturowego związanego zagadnieniem uczenia maszynowego oraz z podejściem low-code/no-code.\ Dzięki temu wybrano narzędzie Machine Learning Studio znajdujące się na platformie Microsoft Azure. \ Microsoft wyszedł naprzeciw potrzebom użytkowników, przygotowując zestaw prekonfigurowanych algorytmów klasyfikacyjnych.\ Możliwości narzędzia Azure ML pozwalają na tworzenie wysokoskalowalnych rozwiązań z zakresu uczenia maszynowego przy relatywnie niskich kosztach.\ Jednakże mogą wystąpić specyficzne wymagania biznesowe albo prawne, które nie będą pozwalały na zastosowanie zewnętrznych narzędzi chmurowych.\ Takim przykładem są strategiczne dane państwowe, których wyciek za granicą może grozić poważnym zagrożeniem z zewnątrz.\ W przypadku systemów wykrywania intruzów zasadne jest stosowanie autorskich rozwiązań, które nie są znane opinii publicznej.\ Takie działanie chroni przed nieautoryzowanym dostępem do sieci komputerowej oraz do danych wrażliwych.
\\ \\
Konkurencyjność rozwiązań autorskich przedstawiono na \refsource{wykresie}{fig:predict-result}.\ Wykorzystanie połączenia algorytmu genetycznego i klasyfikatora naiwnego Bayesa z rozkładem normalnym pozwala na uzyskanie trochę słabszych wyników do algorytmów utworzonych przez Microsoft.\ Różnica w zależności od próby wyniosła od 11 p.p. do 16 p.p. między najlepszym algorytmem a algorytmem \textit{Gaussian Naive Bayes - with GA}.\ Ukazuje to niewielką różnicę w jakości algorytmu.\ Wadą autorskich rozwiązań jest pracochłonność i kosztowność.\ Wiąże się to często z powołaniem i utrzymaniem zespołu projektowego skupionego wobec próby utworzenia konkretnego rozwiązania biznesowego związanego z klasyfikacją danych.\ Koszty takiego przedsięwzięcia będą dużo droższe niż utworzenie podstawowego rozwiązania z wykorzystaniem platformy Azure Machine Learning.\ Koszty jakie zostały poniesione podczas tworzenia potoku zadań i dopracowywaniem go w celu utworzenia środowiska badawczego do niniejszej pracy magisterskiej wyniosły w przybliżeniu około 100\$.\ Wdrożenie takiego rozwiązanie może okazać się tańsze na początek i wystarczające niż posiadanie całego zespołu zajmującego się klasyfikacją i sztuczną inteligencją.
\\ \\
Korzystanie z prostych rozwiązań autorskich również posiada kilka zalet.\ Pozwala to na prototypownie wstępnych rozwiązań biznesowych opartych o klasyfikację danych.\ Zastosowanie algorytmu \textit{GAGNB} nie wymaga wcześniejszej znajomości zbioru danych.\ Pozwala na lokalne korzystanie z programu do klasyfikacji, bez ponoszenia kosztów wykorzystania platformy chmurowej.\ Kolejnym atutem tego algorytmu jest zmniejszenie kosztów lokalnego użytkowania.\ Co zostało spowodowane zmniejszeniem wymiarowości zbioru danych do klasyfikacji poprzez wykorzystanie jedynie wytypowanych kolumn.
\\ \\
Stworzony projekt jest jedynie silnikiem klasyfikacyjnym, który pozwala na wytrenowanie i wyłonienie najlepszego algorytmu do klasyfikacji danych.\ Dzięki możliwościom platformy Azure ML stworzenie całego środowiska testowego jest relatywnie tanie i nie wymaga wielu wyspecjalizowanych umiejętności.\ Bazując na wynikach i najlepszym klasyfikatorze, można utworzyć odpowiednie przepływy służące do klasyfikacji danych wejściowych.\ Dostęp do nich może odbywać się za pomocą utworzonych Punktów Dostępowych \trans{ang. Endpoints}.\ Dzięki punktom dostępowym możliwa jest komunikacja za pomocą protokołu HTTPS \trans{ang. Hyper Text Transfer Protocol Secure} i komunikacja typu REST \textit{(ang. Representative State Transfer)}.
\\ \\
Utworzony w ten sposób punkt dostępu może zostać wykorzystany w klasyfikacji ruchu sieciowego w niewielkich aplikacjach z dostępem do internetu.\ Pozwoliłoby to na realizację analizy danych w chmurze, co mogłoby przyspieszyć cały proces oraz utworzyć pojedyncze źródło prawdy dla wielu instancji aplikacji.\ A wykorzystanie dodatkowo konteneryzacji, którą zapewnia platforma Azure, cały proces mógłby zostać zoptymalizowany pod kątem wydajnościowym i lokalizacyjnym.\ Pojedyncze źródło prawdy jest zaletą wykorzystania ,,\textit{zewnętrznego}'' klasyfikatora, ponieważ zapewnia jednakowe wyniki klasyfikacji w poszczególnych instancjach samej aplikacji instalowanej na wielu urządzeniach.\ Pozwala to również na lepsze dostrajanie całego procesu, a wykorzystanie kolejnych wersji przepływów umożliwia utrzymywanie kopii zapasowych poszczególnych rozwiązań.\ Umożliwia to na przykład cofnięcie wersji klasyfikatora w przypadku wykrycia nieprawidłowości w obecnym modelu.
