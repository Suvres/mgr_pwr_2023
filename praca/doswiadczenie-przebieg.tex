\chapter{Przebieg eksperymentu}
Doświadczenie polegało na analizie porównawczej sprawności algorytmów opisanych w \refsource{rozdziale}{cha:dos} w \refsource{sekcji}{sec:alg}.\ Celem doświadczenia było określenie jakości algorytmu utworzonego w ramach pracy inżynierskiej autora~\cite{Blyszcz2022}.\ Szczegółowa metodologia badawcza została określona w \refsource{podrozdziale}{sec:met}.


\section{Metodologia badawcza}
\label{sec:met}
Przyjęta w projekcie metodologia badawcza została określona w poniższej \refsource{tabeli}{tab:met-bad}.\ Przyjęta metodologia ma za zadanie określić jakoś porównywanego algorytmu.

\begin{table}[H]
    \centering
    \captionsource{Metodologia badawcza}{Opracowanie własne}
    \begin{tabular}{|L{\textwidth}|}
        \hline
        \textbf{Problem badawczy:}                                                                                                                     \\
        Czy algorytm klasyfikacji danych utworzony w ramach pracy inżynierskiej może konkurować z rozwiązaniami dostępnymi w środowiskach komercyjnych \\ \hline

        \textbf{Pytania badawcze:}                                                                                                                     \\
        \begin{enumerate}
            \item Czy algorytm jest konkurencyjny pod względem wybranych metry:
            \begin{itemize}
                \item dokładność algorytmu
                \item czas działania
                \item precyzja
                \item czułość
                \item f1
                \item auc
            \end{itemize}
        \end{enumerate}                                                                                                                                \\ \hline

        \textbf{Hipotezy:}                                                                                                                             \\
        \begin{enumerate}
            \item Nie ma istotnej różnicy pomiędzy wynikami próby testowej i treningowej.
            \item Nie ma istotnej różnicy pomiędzy wynikami prób testowych.
            \item Wynik dopasowania algorytmów nie przekracza dolnej granicy przedziału ufności dla próby testowej
        \end{enumerate}                                                                                                                                \\ \hline
    \end{tabular}
    \label{tab:met-bad}
\end{table}

\vfill
\pagebreak

\section{Przygotowanie platformy badawczej}
Do badań wykorzystano narzędzie ,,\textit{Projektant}'' znajdujące się na platformie ,,\textit{Azure Machine Learning Studio}'' (Azure ML).\ Narzędzie to umożliwiło utworzenie interaktywnego potoku zadań.\ Potok ten składa się z kilku części:
\begin{itemize}
    \item Przygotowanie i obróbka zbiorów danych
    \item Trenowanie oraz testowanie algorytmów klasyfikacji danych
    \item Utworzenie tabeli porównawczej dla wyników poszczególnych algorytmów (\refsource{obraz}{fig:pipeline}).
\end{itemize}

\subsection{Przygotowanie danych}
W pierwszym kroku dane zostały znormalizowane za pomocą metody \textbf{MinMax}, która przekształca dane numeryczne do wartości w zakresie ${0, 1}$.\ W następnym kroku za pomocą języka Python oraz biblioteką Pandas oraz Numpy zostają zamienione etykiety słowne na wartości \textbf{0} i \textbf{1} oraz następuje zamiana wartości $[NaN, -inf, inf]$ na cyfrę $0$.\ Cały proces został zobrazowany na \refsource{diagramie}{fig:norm}

\begin{figure}[H]
    \centering
    \includegraphics[width=0.6\textwidth]{images/norm}
    \captionsource{Potok normalizacji danych}{Opracowanie własne}
    \label{fig:norm}
\end{figure}

\vfill
\pagebreak

\subsection{Trenowanie oraz testowanie algorytmów}
Kolejną grupą zadań widoczną w potoku są te związane z trenowaniem i testowaniem poszczególnych algorytmów opisanych w \refsource{rozdziale}{cha:dos}.\ Każdy test składa się 3 kafelek.\ W przypadku algorytmów dostarczonych wraz z platformą Azure ML są to:
\begin{itemize}
    \item \textbf{model klasyfikujący} - odpowiada za przygotowanie algorytmu klasyfikacyjnego
    \item \textbf{blok treningowy} - tworzy wytrenowany model, za pomocą połączonego zbioru danych
    \item \textbf{blok ewaluacyjny} - sprawdza wcześniej wytrenowany model za pomocą powiązanego zbioru danych.
\end{itemize}
Potok zadań wykorzystujący algorytmy dostarczone przez Microsoft Azure został ukazany na \refsource{schemacie}{fig:ms-pipe}

\begin{figure}[H]
    \centering
    \includegraphics[width=\textwidth]{images/ms_pipe}
    \captionsource{Potok zadań dla algorytmów klasyfikacyjnych}{Opracowanie własne}
    \label{fig:ms-pipe}
\end{figure}

Algorytmy dostarczone w ramach pracy badawczej składają się z:
\begin{itemize}
    \item \textbf{biblioteka Python} - archiwum o rozszerzeniu \textbf{.zip}, które zawiera w sobie odpowiednie pliki napisane w języku Python
    \item \textbf{blok treningowy} - wykorzystuje dostarczoną bibliotekę do wytrenowania modelu oraz zapisania na platformie Azure najlepszego uzyskanego wyniku za pomocą powiązanego zbioru danych
    \item \textbf{blok ewaluacyjny} - wykorzystuje dostarczoną bibliotekę do ewaluacji algorytmu za pomocą połączonego zbioru danych
\end{itemize}

Potok zadań dla algorytmów niestandardowych został ukazany na \refsource{rysunku}{fig:au-pipe}

\begin{figure}[H]
    \centering
    \includegraphics[width=0.8\textwidth]{images/au-pipe}
    \captionsource{Potok zadań dla algorytmów klasyfikacyjnych}{Opracowanie własne}
    \label{fig:au-pipe}
\end{figure}

\subsection{Utworzenie tabeli porównawczej}
Kolejną częścią zadań jest zebranie wyników poszczególnych algorytmów oraz połączenie ich w jedną całość.\ Wykorzystano do tego moduły języka Pyton, które zwracają przetworzone wyniki oraz łączą je w jedną tabelę zbiorczą, co pokazano na \refsource{rysunku}{fig:pipe-4}.

\begin{figure}[H]
    \centering
    \includegraphics[width=\textwidth]{images/pipe-csv}
    \captionsource{Moduły odpowiedzialne za przetwożenie wyników}{Opracowanie własne}
    \label{fig:pipe-4}
\end{figure}


\section{Weryfikacja potoku}
Aby zweryfikować działanie całego procesu wykorzystano znormalizowane dane treningowe do wytrenowania oraz przetestowania działania algorytmów klasyfikacyjnych.\ Cały proces trwał ,,\textbf{1 dzień 10 godzin 55 minut 53 sekundy}''.\ Wyniki tych działań widać na \refsource{rysunku}{fig:predict-same}.\ Analizując wykres można zauważyć, że uzyskane wyniki znajdują się w przedziale $[94\%, 100\%]$ w każdej metryce co pokazuje jakość każdego z algorytmów, a także to, że algorytmy poradziły sobie niemal bezbłędnie w rozpoznawaniu ruchu sieciowego, na którym były uczone.\ Zbiór, który wykorzystano do trenowania oraz testowania danych zawierał w sobie 225805 wpisów z czego 97718 należało do klasy ,,\textbf{1}'', zaś 128087 należało do klasy ,,\textbf{0}''.

\begin{table}[H]
    \centering
    \captionsource{Liczba elementów przynależących do danej klasy w zniorze treningowym}{Opracowanie własne}
    \label{tab:trening-data-label}
    \begin{tabular}{|c|r|}
        \hline
        \textbf{Klasa} & \textbf{Liczba wystąpień} \\ \hline
        1              & 97718                     \\ \hline
        0              & 128027                    \\ \hline
        \textbf{Suma}  & 225805                    \\ \hline
    \end{tabular}
\end{table}

Bazując na tym zbiorze oraz uzyskanych wynikach udało się udowodnić poprawność działania procesu klasyfikacji wieloma algorytmami genetycznymi.

\begin{landscape}
    \vspace*{\fill}
    \begin{figure}[H]
        \centering
        \includegraphics[height=0.8\textwidth]{images/predict_same}
        \captionsource{Wyniki testów algorytmów klasyfikacyjnych na danych treningowych}{Opracowanie własne}
        \label{fig:predict-same}
    \end{figure}
    \vfill
\end{landscape}


\section{Próba badawcza}
Aby uzyskać realne wyniki podczas porównywania poszczególnych algorytmów zastosowano zbiór treningowy opisany w \refsource{tabeli}{tab:trening-data-label} oraz zbiór testowy, który zawierał 2273097 wpisów należących do klasy ,,\textbf{1}'' oraz 557646 wpisów należących do klasy ,,\textbf{0}''.\ Sumarycznie ilość wpisów wynosi: 2830743, co zostało pokazane w \refsource{tabel}{tab:res-test}.\ Pomiary testowe powtórzono 2 razy dzięki czemu uzyskano 3 próby badawcze.

\begin{table}[H]
    \centering
    \captionsource{Liczba elementów przynależących do danej klasy w zbiorze testowym}{Opracowanie własne}
    \label{tab:res-test}
    \begin{tabular}{|c|r|}
        \hline
        \textbf{Klasa} & \textbf{Liczba wystąpień} \\ \hline
        1              & 2273097                   \\ \hline
        0              & 557646                    \\ \hline
        \textbf{Suma}  & 2830743                   \\ \hline
    \end{tabular}
\end{table}

Poniżej zostały przedstawione wyniki zbiorcze dla poszczególnych metryk.\ Dodatkowo przedstawiono również wynik pomiaru treningowego, który w większości przypadków jest wyższy od danych testowych.\ Co prawdopodobnie jest spowodowane różnicą w ilości danych testowych i treningowych.\ Dodatkowo w każdej kolumnie oznaczono kolorem zielonym najwyższy wynik dla danej metryki, a kolorem czerwonym najniższy wynik dla danej metryki.

\begin{landscape}
    \vspace*{\fill}
    \begin{figure}[H]
        \centering
        \includegraphics[height=0.8\textwidth]{images/predict_result}
        \captionsource{Wyniki testów algorytmów klasyfikacyjnych na danych testowych}{Opracowanie własne}
        \label{fig:predict-result}
    \end{figure}
    \vfill
\end{landscape}

\subsection{Wyniki dopasowania}
Najlepszy wynik dopasowania dla danych testowych uzyskał algorytm \textit{Two-Class Average Perceptron}, który poprawnie rozpoznał $85,7768\%$ próbek.\ Najgorszy wynik uzyskał algorytm \textit{DANet} z dopasowaniem rzędu: $19,9691\%$.\ Dla próby treningowej najlepszy wynik uzyskał \textit{Two-Class Boosted Decision Tree} z wynikiem $100,00\%$, a najgorszy \textit{Two-Class Average Perceptron} z wynikiem $97,4502\%$.\ Wyniki dopasowania dla poszczególnych prób zostały przedstawione na \refsource{tabeli}{tab:acc-res} oraz na \refsource{wykresie}{fig:acc-res}.

\begin{table}[H]
    \centering
    \captionsourceb{Wynik dopasowania algorytmów.}{Kolorem zielonym określono najlepszy wynik w kolumnie.\ Kolorem czerwonym określono najgorszy wynik w kolumnie.}{Opracowanie własne}
    \resizebox{\textwidth}{!}{%
    \begin{NiceTabular}{|l|r|r|r||r|}[hvlines]
        & \multicolumn{4}{c|}{\textbf{Wynik dopasowania}} \\
        \textbf{Algorytm}                & \textbf{Próba 1}                  & \textbf{Próba 2}                  & \textbf{Próba 3}                  & \textbf{Próba testowa}             \\
        Two-Class Support Vector Machine & $80,3002\%$                       & $80,3002\%$                       & $80,3002\%$                       & $99,9167\%$                        \\
        Two-Class Boosted Decision Tree  & $83,8482\%$                       & $83,8482\%$                       & $83,8482\%$                       & \cellcolor{lightgreen}$100,0000\%$ \\
        Two-Class Decision Forest        & $83,1725\%$                       & $83,1725\%$                       & $83,1725\%$                       & $99,9960\%$                        \\
        Two-Class Neural Network         & $84,7332\%$                       & $84,7332\%$                       & $84,7332\%$                       & $99,2682\%$                        \\
        Two-Class Average Perceptron     & \cellcolor{lightgreen}$85,7768\%$ & \cellcolor{lightgreen}$85,7768\%$      & \cellcolor{lightgreen}$85,7768\%$       & \cellcolor{lightred}$97,4502\%$             \\
        Gaussian Naive Base - with GA    & $80,3004\%$                       & $80,3004\%$                       & $80,3004\%$                       & $98,8013\%$                        \\
        DANet                            & \cellcolor{lightred}$19,9691\%$   & \cellcolor{lightred}$19,7899\%$   & \cellcolor{lightred}$19,7899\%$      & $99,9823\%$            \\
    \end{NiceTabular}%
    }
    \label{tab:acc-res}
\end{table}

\begin{figure}[H]
    \centering
    \includegraphics[width=\textwidth]{images/acc-res}
    \captionsource{Dokładność algorytmów}{Opracowanie własne}
    \label{fig:acc-res}
\end{figure}

\subsection{Wyniki precyzji}
Najlepszy wynik precyzji dla danych testowych uzyskał algorytm \textit{Two-Class Neural Network} ($85,9776\%$).\ Najgorszy wynik uzyskał algorytm \textit{Two-Class Support Vector Machine} ($80,3003\%$).\ Dla próby treningowej najlepszy wynik uzyskał \textit{Two-Class Boosted Decision Tree} ($100\%$), a najgorszy \textit{Two-Class Average Perceptron} ($94,5\%$).\ Wyniki precyzji dla poszczególnych prób zostały przedstawione w \refsource{tabeli}{tab:acc-prec} oraz na wykresie \refsource{wykresie}{fig:prec-res}.

\begin{table}[H]
    \centering
    \captionsourceb{Wynik precyzji algorytmów.}{Kolorem zielonym określono najlepszy wynik w kolumnie.\ Kolorem czerwonym określono najgorszy wynik w kolumnie.}{Opracowanie własne}
    \resizebox{\textwidth}{!}{%
    \begin{NiceTabular}{|l|r|r|r||r|}[hvlines]
        \hline
        & \multicolumn{4}{c|}{\textbf{Wynik precyzji}} \\
        \textbf{Algorytm}                & \textbf{Próba 1}                  & \textbf{Próba 2}                  & \textbf{Próba 3}                  & \textbf{Próba testowa}             \\
        Two-Class Support Vector Machine & \cellcolor{lightred}$80,3003\%$   & \cellcolor{lightred}$80,3003\%$   & \cellcolor{lightred}$80,3003\%$   & $99,8742\%$            \\
        Two-Class Boosted Decision Tree  & $83,5016\%$                       & $83,5016\%$                       & $83,5016\%$                       & \cellcolor{lightgreen}$100,0000\%$ \\
        Two-Class Decision Forest        & $82,6752\%$                       & $82,6752\%$                       & $82,6752\%$                       & $99,9908\%$                        \\
        Two-Class Neural Network         & \cellcolor{lightgreen}$85,9776\%$ & \cellcolor{lightgreen}$85,9776\%$ & \cellcolor{lightgreen}$85,9776\%$ & $99,8972\%$            \\
        Two-Class Average Perceptron     & $84,9806\%$                       & $84,9806\%$                       & $84,9806\%$                       & \cellcolor{lightred}$94,5000\%$    \\
        Gaussian Naive Base - with GA    & $85,7768\%$                       & $80,3004\%$                       & $80,3004\%$                       & $99,8531\%$                        \\
        DANet                            & $82,7536\%$                       & $85,9516\%$                       & $85,9516\%$                       & $99,9887\%$                        \\
    \end{NiceTabular}%
    }
    \label{tab:acc-prec}
\end{table}

\begin{figure}[H]
    \centering
    \includegraphics[width=\textwidth]{images/prec-res}
    \captionsource{Precyzja algorytmów}{Opracowanie własne}
    \label{fig:prec-res}
\end{figure}

\subsection{Wyniki czułości}
Najlepszy wynik czułości dla danych testowych uzyskał algorytm \textit{Gaussian Naive Base - with GA} ($100,00\%$).\ Najgorszy wynik uzyskał algorytm \textit{DANet} ($0,4239\%$).\ Dla próby treningowej najlepszy wynik uzyskał \textit{Two-Class Boosted Decision Tree} oraz \textit{Two-Class Decision Forest} ($100\%$), a najgorszy \textit{Gaussian Naive Base - with GA} ($97,3741\%$).\ Wyniki czułości dla poszczególnych prób zostały przedstawione w \refsource{tabeli}{tab:acc-rec} oraz na wykresie \refsource{wykresie}{fig:rec-res}.

\begin{table}[H]
    \centering
    \captionsourceb{Wynik czułości algorytmów.}{Kolorem zielonym określono najlepszy wynik w kolumnie.\ Kolorem czerwonym określono najgorszy wynik w kolumnie.}{Opracowanie własne}
    \resizebox{\textwidth}{!}{%
    \begin{NiceTabular}{|l|r|r|r||r|}[hvlines]

        & \multicolumn{4}{c|}{\textbf{Wynik czułości}} \\
        \textbf{Algorytm}                & \textbf{Próba 1}                   & \textbf{Próba 2}                   & \textbf{Próba 3}                   & \textbf{Próba testowa}             \\
        Two-Class Support Vector Machine & $99,9998\%$                        & $99,9998\%$                        & $99,9998\%$                        & $99,9335\%$                        \\
        Two-Class Boosted Decision Tree  & $99,5563\%$                        & $99,5563\%$                        & $99,5563\%$                        & \cellcolor{lightgreen}$100,0000\%$ \\
        Two-Class Decision Forest        & $99,9996\%$                        & $99,9996\%$                        & $99,9996\%$                        & \cellcolor{lightgreen}$100,0000\%$ \\
        Two-Class Neural Network         & $96,7705\%$                        & $96,7705\%$                        & $96,7705\%$                        & $98,4107\%$                        \\
        Two-Class Average Perceptron     & $99,9531\%$                        & $99,9531\%$                        & $99,9531\%$                        & $99,9253\%$                        \\
        Gaussian Naive Base - with GA    & \cellcolor{lightgreen}$100,0000\%$ & \cellcolor{lightgreen}$100,0000\%$ & \cellcolor{lightgreen}$100,0000\%$ & \cellcolor{lightred}$97,3741\%$            \\
        DANet                            & \cellcolor{lightred}$0,4239\%$     & \cellcolor{lightred}$0,1343\%$     & \cellcolor{lightred}$0,1343\%$     & $99,9703\%$            \\
    \end{NiceTabular}%
    }
    \label{tab:acc-rec}
\end{table}

\begin{figure}[H]
    \centering
    \includegraphics[width=\textwidth]{images/rec-res}
    \captionsource{Czułość algorytmów}{Opracowanie własne}
    \label{fig:rec-res}
\end{figure}

\subsection{Wyniki f1}
Najlepszy wynik F1 dla danych testowych uzyskał algorytm \textit{Two-Class Average Perceptron} ($91,8606\%$).\ Najgorszy wynik uzyskał algorytm \textit{DANet} ($0,8434\%$).\ Dla próby treningowej najlepszy wynik uzyskał \textit{Two-Class Boosted Decision Tree} ($100\%$), a najgorszy \textit{Two-Class Average Perceptron} ($97,1370\%$).\ Wyniki precyzji dla poszczególnych prób zostały przedstawione w \refsource{tabeli}{tab:acc-f1} oraz na wykresie \refsource{wykresie}{fig:f1-res}.

\begin{table}[H]
    \centering
    \captionsourceb{Wynik F1 algorytmów.}{Kolorem zielonym określono najlepszy wynik w kolumnie.\ Kolorem czerwonym określono najgorszy wynik w kolumnie.}{Opracowanie własne}
    \resizebox{\textwidth}{!}{%
    \begin{NiceTabular}{|l|r|r|r||r|}[hvlines]

        & \multicolumn{4}{c|}{\textbf{Wynik F1}} \\
        \textbf{Algorytm}                & \textbf{Próba 1}                  & \textbf{Próba 2}                  & \textbf{Próba 3}                  & \textbf{Próba testowa}             \\
        Two-Class Support Vector Machine & $89,0739\%$                       & $89,0739\%$                       & $89,0739\%$                       & $99,9038\%$                        \\
        Two-Class Boosted Decision Tree  & $90,8249\%$                       & $90,8249\%$                       & $90,8249\%$                       & \cellcolor{lightgreen}$100,0000\%$ \\
        Two-Class Decision Forest        & $90,5159\%$                       & $90,5159\%$                       & $90,5159\%$                       & $99,9954\%$                        \\
        Two-Class Neural Network         & $91,0553\%$                       & $91,0553\%$                       & $91,0553\%$                       & $99,1484\%$                        \\
        Two-Class Average Perceptron     & \cellcolor{lightgreen}$91,8608\%$ & \cellcolor{lightgreen}$91,8608\%$ & \cellcolor{lightgreen}$91,8608\%$ & \cellcolor{lightred}$97,1370\%$            \\
        Gaussian Naive Base - with GA    & $89,0740\%$                       & $89,0740\%$                       & $89,0740\%$                       & $98,5980\%$                        \\
        DANet                            & \cellcolor{lightred}$0,8434\%$    & \cellcolor{lightred}$0,2682\%$    & \cellcolor{lightred}$0,2682\%$    & $99,9795\%$            \\
    \end{NiceTabular}%
    }
    \label{tab:acc-f1}
\end{table}

\begin{figure}[H]
    \centering
    \includegraphics[width=\textwidth]{images/f1-res}
    \captionsource{F1 algorytmów}{Opracowanie własne}
    \label{fig:f1-res}
\end{figure}

\subsection{Wyniki AUC}
Najlepszy wynik precyzji dla danych testowych uzyskał algorytm \textit{Two-Class AveragePerceptron} ($79,1892\%$).\ Najgorszy wynik uzyskał algorytm \textit{Two-Class Support Vector Machine} ($49,999\%$).\ Dla próby treningowej najlepszy wynik uzyskał \textit{Two-Class Boosted Decision Tree} oraz \textit{Two-Class Decision Forest} ($100\%$), a najgorszy \textit{Gaussiaon Naive Base - with GA} ($99,9437\%$).\ Wyniki precyzji dla poszczególnych prób zostały przedstawione w \refsource{tabeli}{tab:acc-auc} oraz na wykresie \refsource{wykresie}{fig:auc-res}.

\begin{table}[H]
    \centering
    \captionsourceb{Wynik AUC algorytmów.}{Kolorem zielonym określono najlepszy wynik w kolumnie.\ Kolorem czerwonym określono najgorszy wynik w kolumnie.}{Opracowanie własne}
    \resizebox{\textwidth}{!}{%
    \begin{NiceTabular}{|l|r|r|r||r|}[hvlines]

        & \multicolumn{4}{c|}{\textbf{Wynik AUC}} \\
        \textbf{Algorytm}                & \textbf{Próba 1}                  & \textbf{Próba 2}                  & \textbf{Próba 3}                  & \textbf{Próba testowa} \\
        Two-Class Support Vector Machine & \cellcolor{lightred}$49,9999\%$   & \cellcolor{lightred}$49,9999\%$ & \cellcolor{lightred}$49,9999\%$     & $99,9974\%$            \\
        Two-Class Boosted Decision Tree  & $69,9089\%$                       & $69,9089\%$                       & $69,9089\%$                       & \cellcolor{lightgreen}$100,0000\%$           \\
        Two-Class Decision Forest        & $63,5244\%$                       & $63,5244\%$                       & $63,5244\%$                       & \cellcolor{lightgreen}$100,0000\%$           \\
        Two-Class Neural Network         & $72,1864\%$                       & $72,1864\%$                       & $72,1864\%$                       & $99,9867\%$            \\
        Two-Class Average Perceptron     & \cellcolor{lightgreen}$79,1892\%$ & \cellcolor{lightgreen}$79,1892\%$ & \cellcolor{lightgreen}$79,1892\%$ & $99,9808\%$            \\
        Gaussian Naive Base - with GA    & $50,0000\%$                       & $50,0000\%$                       & $50,0000\%$                       & \cellcolor{lightred}$99,9437\%$            \\
        DANet                            & $50,4435\%$                       & $76,7282\%$                       & $76,7282\%$                       & $99,9993\%$            \\
    \end{NiceTabular}%
    }
    \label{tab:acc-auc}
\end{table}

\begin{figure}[H]
    \centering
    \includegraphics[width=\textwidth]{images/auc-res}
    \captionsource{AUC algorytmów}{Opracowanie własne}
    \label{fig:auc-res}
\end{figure}


\section{Analiza wyników}
Wszystkie poniższe testy statystyczne zostały wykonane dla założeń z \refsource{tabel}{tab:stat}:

\begin{table}[H]
    \centering
    \captionsource{Założenia wykorzystywane do analizy statystycznej danych}{Opracowanie własne}
    \begin{tabular}{|l|c|} \hline
        \textbf{Założenie} & \textbf{Wartość} \\ \hline
        Przedział ufności & $95\%$ \\ \hline
        $\alpha$ & $0,05$ \\ \hline
        Liczba elementów & $7$ \\ \hline
    \end{tabular}
    \label{tab:stat}
\end{table}

\subsection{Hipoteza $H_0$: Nie ma istotnej różnicy pomiędzy wynikami ''dopasowania'' próby testowej i treningowej}

Wykorzystując statystyczny test t Studenta dla prób zależnych dla danych z \refsource{tabeli}{tab:acc-res} oraz założenia z \refsource{tabeli}{tab:stat} określono, że wartość $p-value = 0,0163$.\ Oznacza to, że zmienna $p-value < \alpha$, dzięki czemu można odrzucić hipotezę $H_0$.\ Wyniki tej analizy określają, że widać istotne różnice pomiędzy danymi z próby testowej i treningowej.\ Największą różnicę widać w rezultacie algorytmu \textit{DANet}, który uzyskał w próbie testowej $19,9691\%$ dopasowania, a w próbie treningowej $99,9823\%$.\\

\subsection{Hipoteza $H_0$: Nie ma istotnej różnicy pomiędzy wynikami ''dopasowania'' prób testowych}

Za pomocą testu t Studenta dla prób zależnych określono porównano dane w próbach testowych z \refsource{tabeli}{tab:acc-res}.\ Uzyskane wyniki, przedstawione w \refsource{tabeli}{tab:acc-p-stat}, pozwalają zachować Hipotezę $H_0$, stwierdzającą, że pomiędzy danymi w poszczególnych próbach testowych nie ma istotnych różnic.

\begin{table}[H]
    \centering
    \captionsource{Wyniki testu t Studenta dla poszczególnych prób testowych}{Opracowanie własne}
   \begin{tabular}{|c|c|c|} \hline
       \multicolumn{3}{|c|}{$H_0$ Brak istotnych różnic między wynikami dla $\alpha = 0,05$} \\ \hline
       \textbf{Relacja} & \textbf{P-value} & \textbf{Rezultat} \\\hline
       Próba 2 $\leftrightarrow$ Próba 3 & $0,5 > \alpha$ & Brak istotnych różnic \\ \hline
       Próba 2 $\leftrightarrow$ Próba 1 & $0,5 > \alpha$ & Brak istotnych różnic \\ \hline
       Próba 3 $\leftrightarrow$ Próba 1 & $0,5 > \alpha$ & Brak istotnych różnic \\ \hline
   \end{tabular}
    \label{tab:acc-p-stat}
\end{table}

\subsection{Hipoteza $H_0$: Wynik dopasowania algorytmów nie przekracza dolnej granicy przedziału ufności}


Przedział ufności dla dokładności algorytmów wyniósł $17,7218$ dla $\alpha = 0.05$.\ Biorąc pod uwagę ten fakt można stwierdzić, że dokładność algorytmu DANet jest poniżej dolnej granicy.\ Dolna granica przedziału ufności wynosi $56,2925\%$, zaś DANet uzyskał $19,9691\%$ oraz poprawnie zaklasyfikował jedynie $565273$ wpisów.\ Oznacza to, że można odrzucić $H_0$, ponieważ jeden algorytm przekracza dolny próg granicy ufności


\section{Wnioski}

Microsoft Wyszedł naprzeciw potrzebom użytkowników przygotowując zestaw prekonfigurowanych algorytmów klasyfikacyjnych.\ Możliwości narzędzia Azure ML pozwalają na odpowiadanie na konkretne potrzeby przy relatywnie niewielkich kosztach.\ To nie oznacza jednak, że tworzenie autorskich rozwiązań mija się z celem.\ Jak ukazano na \refsource{wykresie}{fig:predict-result} autorskie rozwiązania również mają rację bytu.\ Wykorzystanie połączenia algorytmu genetycznego i klasyfikatora naiwnego Bayesa z rozkładem normalnym pozwala na uzyskanie zbliżonych wyników do algorytmu utworzone przez giganta technologicznego.\ Różnica około 5 punktów procentowych między najlepszym algorytmem a algorytmem GAGNB ukazuje niewielką różnicę w jakości algorytmu.\ Dodatkowo wciąż trudno korzysta się z rozwiązań takich jak DANet, które są nieprzetestowane na innych zbiorach niż tych przygotowanych z algorytmem.
\\ \\
Dodatkowo korzystanie z tego typu prostych rozwiązań autorskich pozwala na prototypownie rozwiązań biznesowych opartych o klasyfikację danych.\ Samo wykorzystanie algorytmu genetycznego z algorytmem GNB umożliwia skupienie się na tworzenie ogólnego rozwiązania.\ Nie wymaga to wcześniejszej znajomości zbioru oraz pozwala na korzystanie z programów klasyfikacyjnych lokalnie nie ponosząc kosztów wykorzystania platformy chmurowej.\ Kolejnym atutem utworzonego przez autora rozwiązania jest zmniejszenie kosztów lokalnego użytkowania.\ Co zostało spowodowane zmniejszeniem wymiarowości zbioru danych do klasyfikacji poprzez wykorzystanie jedynie wytypowanych kolumn.
\\ \\
Doświadczenie to ukazuje, że wciąż należy próbować tworzyć wydajniejsze i dokładniejsze rozwiązania, lecz nie zaprzecza faktu iż rozwiązania ogólnie dostępne są na bardzo wysokim poziomie.\ Uzyskano dokładność z zakresu $[80\%, 85\%]$ przy czym plik testowy był 12 krotnie większy od pliku treningowego.

